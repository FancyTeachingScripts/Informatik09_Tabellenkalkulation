\Hefteintrag[40]{1}{Wenn-Dann-Funktion}{

\vspace{0.5cm}
\begin{enumerate}
    \item Öffne Studyflix: \UrlAndCode[-1.5cm][-0.85cm]{bycs.link/studyflix-excel-if}
    \item Schaue das Video und baue die beschriebene Tabelle nach.
    \item Fasst den Artikel/das Video in einem kurzen \emphColA{Hefteintrag} zusammen.
\end{enumerate}
\LoesungKaro{

\ifbeamer
\vspace{0.2cm}
    \begin{minipage}{0.8\textwidth}
\fi
Mit der  \emphColA{Wenn-Dann-Funktion} können anhand einer Bedingung verschiedene Werte verwendet werden. 

\vspace{0.2cm}
Eine Bedingung kann z.B. 
\begin{itemize}
    \item \emphColA{Gleichheit zweier Werte (=)} oder
    \item eine \emphColB{Größer-/Kleiner-Bedingung (<,>,<=,>=)}
\end{itemize} 
prüfen. 

\vspace{0.2cm}
Wenn die \emphColA{Bedingung} als \emphColA{wahr} ausgewertet \emphColA{(=erfüllt)} wird, wird der \emphColB{Dann-Teil in die Zelle eingefügt}, ansonsten der \emphColC{Sonst-Teil}.

\vspace{0.5cm}
In Excel gibt man die Funktion so ein:

\vspace{0.2cm}
Schema:~~~~=WENN(\emphColA{Bedingung}  ; \emphColB{Dann} ; \emphColC{Sonst})

Beispiel:~~~~=WENN(\emphColA{D5 < 10} ; \emphColB{„kleiner als 10“} ; \emphColC{„größer oder gleich 10“})

\ifbeamer
\end{minipage}
\fi
}{43}
}