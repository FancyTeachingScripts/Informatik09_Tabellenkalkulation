    \section{So viel zu beachten...}
    Bei einer Party im geplanten Maßstab fallen nicht nur Getränkekosten an. Zeichne zu jedem der folgenden Posten jeweils ein Datenflussdiagramm (du darfst verständliche Abkürzungen verwenden)\\\\
    
    \large\textbf{Getränkeeinkauf} \normalsize \hspace{0.2cm}
    Die Kosten für alle Getränke zusammen berechnest du aus der Anzahl an Personen, der Anzahl an Getränken pro Person und den Kosten pro einzelnem Getränk. Der Einfachheit halber gehen wir davon aus, dass jedes Getränk den selben Preis hat.\\
    \LoesungKaro{}{9}\\\\
    
    \large\textbf{Getränkegewinn} \normalsize \hspace{0.2cm}
    Durch den Verkauf der Getränke nimmst du Geld ein. Am Ende der Party zählst du die Kassen und erhältst die Gesamteinnahmen. Aus diesem Betrag und den Ausgaben errechnest du den Gewinn.\\
    \LoesungKaro{}{9}\\\\
    
    \large\textbf{Security} \normalsize \hspace{0.2cm}
    Weil die Feier deiner besten Freundin beim letzten Mal eskaliert ist, engagierst du einen Sicherheitsdienst. Die Anzahl der benötigten Security-Mitarbeiter berechnest du aus der Anzahl an Gästen und einem Personenschlüssel. Im Anschluss werden aus der Anzahl an Mitarbeitern und den Kosten pro Mitarbeiter die Security-Kosten berechnet.\\
    \LoesungKaro{}{12}\\
    
    \newpage
    
    \large\textbf{Fixkosten} \normalsize \hspace{0.2cm}
    Da in deinem Wohnzimmer nicht genügend Platz ist, mietest du eine Partylocation. Aus der Miete und der Gage für die Band ergeben sich die Fixkosten.\\
    \LoesungKaro{}{9}\\
    
    \section{It's coming all together}
    Aus allen vorherigen Einzelposten errechnet du die Gesamtkosten deiner Party. Füge die einzelnen Datenflussdiagramm zu einem großen Gesamtdiagramm zusammen. Ein Datenblock darf als Eingabedaten für mehrere Funktionen verwendet werden.\\
    \LoesungKaro{}{30}