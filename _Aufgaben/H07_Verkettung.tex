\Hefteintrag{2}{Verkettung von Funktionen}{
    Wenn der \LoesungLuecke{Ausgabewert}{6cm} einer Funktion als \LoesungLuecke{Eingabewert}{6cm} einer anderen Funktion verwendet wird, spricht man von \LoesungLuecke{Verkettung}{6cm} von Funktionen. In Datenflussdiagrammen können \LoesungLuecke{Datenblöcke}{7cm} zwischen \LoesungLuecke{2 Funktionen}{10cm} weggelassen werden. Hierbei ist es dann besonders wichtig, aussagekräftige Funktionsnamen zu wählen. Mit einem \LoesungLuecke{Verteiler}{6cm} kann ein 

    \begin{minipage}[t]{0.5\textwidth}
        \vspace{-2cm}
        Datenfluss hierfür in zwei aufgeteilt werden:

        \vspace{12pt}
        Ein \emphColA{Beispiel} ist das Gesamt-Diagramm aus der  \emphColA{vorherigen Aufgabe}.
    \end{minipage}
    \hfill
    \begin{minipage}[t]{0.4\textwidth}
        \LoesungKaroTikz{
            \node[oval, minimum width=0.3cm, minimum height=0.3cm] (circ) {};
            \node[left=1cm of circ] (d1) {};
            \node[above right = 0.5cm and 1cm of circ] (d2) {};
            \node[below right = 0.5cm and 1cm of circ] (d3) {};

            \draw[arrow] (d1)--(circ);
            \draw[arrow] (circ)--(d2);
            \draw[arrow] (circ)--(d3);
        }{5}
        %\vfill
    \end{minipage}
}   