\Aufgabe{Übung: Funktionale Modellierung}{
    Bei einer großen Party fallen nicht nur Getränkekosten an. Zeichne jeweils zwei Datenflussdiagramme: 
    \begin{itemize}
        \item Eines auf höchster Abstraktionsebene für Daten und Funktionen (genau eine Funktion pro Einzel-Diagramm).
        \item Eines mit konkreten Rechenoperationen in Funktionen (2-stellige Funktionen) und Daten auf höchster Abstraktionsebene.
    \end{itemize}
}
\UnterAufgabe{Übung: Funktionale Modellierung (a)}{
    
    \large\textbf{Getränkegewinn} \normalsize \hspace{0.2cm}
    Durch den Verkauf der Getränke nimmst du Geld ein. Am Ende der Party zählst du die Kassen und erhältst die Gesamteinnahmen. Aus diesem Betrag und den Ausgaben beim Lieferanten errechnest du den Gewinn.\\
    \LoesungKaroTikz{
                \node[box, sharp corners, text width=4cm, minimum width=1pt] (werte) {Werte (Eingaben/Ausgaben)};

                \node[oval, text width=2.2cm, right=0.2cm of werte] (fkt) {Funktionen};
                
                \node[minimum width=1pt, below=1cm of werte] (d1) {Datenflüsse: };
                \node[minimum width=1pt, right=1cm of d1] (d2) { };
                \draw[arrow] (d1)--(d2);
        }{16}
}
\UnterAufgabe{Übung: Funktionale Modellierung (b)}{
    \begin{minipage}[t]{\textwidth}
    \large\textbf{Security} \normalsize \hspace{0.2cm}
    Weil die Feier deiner besten Freundin beim letzten Mal eskaliert ist, engagierst du einen Sicherheitsdienst. Die Anzahl der benötigten Security-Mitarbeiter berechnest du aus der Anzahl an Gästen und einem Personenschlüssel. Im Anschluss werden aus der Anzahl an Mitarbeitern und den Kosten pro Mitarbeiter die Security-Kosten berechnet.\\
    \LoesungKaro{}{15}\\
    \end{minipage}
}

\UnterAufgabe{Übung: Funktionale Modellierung (c)}{
    \begin{minipage}[t]{\textwidth}
    \large\textbf{Anzahl Gäste} \normalsize \hspace{0.2cm}
    Du hast vergessen, am Einlass eine Strichliste zu führen, daher kennst du nur deine Einnahmen durch Eintrittskarten und wie viel eine gekostet hat. Hier raus berechnest du die Anzahl der Gäste.
    
    \LoesungKaro{}{13}
    
    \end{minipage}
}

\UnterAufgabe{Übung: Funktionale Modellierung (d)}{
    \begin{minipage}{\textwidth}
    \large\textbf{Gewinn pro Gast} \normalsize \hspace{0.2cm}
    Aus dem Getränke-Gewinn, den Security-Kosten und der Gästeanzahl berechnest du den durchschnittlichen Gewinn pro Gast.
    
    \LoesungKaro{}{15}
    \end{minipage}
}
    
\UnterAufgabe{Übung: Funktionale Modellierung (e)}{
    \begin{minipage}[t]{\textwidth}
    \large\textbf{Gesamt-Diagramm} \normalsize \hspace{0.2cm}
    Füge die vorherigen Einzel-Diagramme zu zwei verketteten Datenflussdiagrammen zusammen. 
    
    \LoesungKaro{}{45}
    \end{minipage}
}