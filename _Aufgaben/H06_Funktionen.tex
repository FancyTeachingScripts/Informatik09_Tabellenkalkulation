\Hefteintrag{2}{Funktionen und Stelligkeit}{
Eine Funktion besitzt in der Informatik genauso wie in Mathe Eingaben (=\LoesungLuecke{Parameter}{4.7cm}) und genau eine Ausgabe (=\LoesungLuecke{Rückgabewert}{6cm}).

Besitzt eine Funktion \emphColA{einen} Parameter heißt sie \LoesungLuecke{einstellig}{6cm}, bei \emphColA{zwei} Parametern \LoesungLuecke{zweistellig}{6cm} usw.

Gewöhnliche \emphColA{Rechenoperationen sind \LoesungLuecke{zweistellige}{6cm} Funktionen}. SUMME und PRODUKT können auch als fertige Funktion geschrieben werden und sind dann \emphColB{beliebig vielstellig}. 

Einzelne \emphColA{Parameter trennt} man mit \emphColA{Semikolon}, alle Zellen innerhalb eines \emphColB{Bereichs} gibt man mit \emphColB{Doppelpunkt zwischen Start- und Endzelle} an. \emphColA{Zum Beispiel:}

\vspace{0.3cm}
\LoesungLuecke{= A1 + B1 + C1 + D1 = SUMME(A1;B1;C1;D1) = SUMME(A1:D1)}{17.5cm}
}