\Aufgabe{Getränkekalkulation}
{
    \AttachVlg{\faFilePdfO}{_Aufgaben/resources/A05_Stationen.pdf}
    Ihr macht die Kalkulation für eine große Party mit einer Kalkulationstabelle. Da so eine Planung aufwendig ist, wird sie auf mehrere Personen aufgeteilt.
    \begin{enumerate}
        \item Bildet mindestens 4 Gruppen (A1,A2,B1,B2 - manche kann es doppelt geben) und nehmt euch gemeinsam einen Zettel. Eure Aufgabenstellung erhaltet ihr von der Lehrkraft \Loesung{(oben als Dateianhang)}
        \item Zeichnet zu eurer Aufgabenstellung \emphColB{pro Berechnungsschritt ein Datenflussdiagramm} (mit hoher Abstraktion)\\
            \hinweis{Zur Erinnerung: hohe Abstraktion bedeutet keine konkreten Rechnungen, sondern beschreibende Funktionsnamen.}\\
            \hinweis{Wenn die gleiche Berechnung für mehrere Getränke gemacht werden muss, zeichnet hierfür auch mehrere Diagramme.}
        \item Tauscht euer Diagramm mit der anderen Gruppe (eures Buchstabens, also z.B. tauschen A1 und A2) und setzt dieses dann mit der Tabellensoftware in BYCS-Drive um. \begin{itemize}
        \item Färbt auch dieses Mal wieder die Zellen anhand des Typs (Nutzereingabe, Formel, Beschriftung) ein.
            \item Zum Testen eurer Formeln könnt ihr einfach Preise und Gäste-Anzahlen erfinden.
            \item Beantwortet außerdem folgende Fragen:
        \end{itemize}
    \end{enumerate}

    Wieso ist es sinnvoll, zuerst ein Diagramm zu zeichnen?

    \LoesungLine{z.B. Besserer Überblick, Aufbau einer Intuition für den Kontext, geringere Gefahr vor lauter Syntax den Überblick zu verlieren, 'Divide-and-Conquer', erst Planen, dann Umsetzen reduziert Fehler}{2}


    Welche Eigenschaften eines Diagramms machen die Umsetzung leichter?

    \LoesungLine{aussagekräftige Namen für Werte auch ohne den Kontext zu kennen, beschreibende Funktionsnamen statt nur Rechenoperationen, \dots}{2}
}