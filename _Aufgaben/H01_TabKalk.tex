\Hefteintrag{1.8}{Tabellenkalkulation}{ In Tabellenkalkulationsprogrammen können Daten in den Zellen der \LoesungLuecke{Tabellenblätter}{6cm} erfasst und mithilfe von \emphColA{Formeln} verarbeitet werden. Jede Zelle besitzt eine eindeutige \emphColA{Adresse}. Diese besteht aus \emphColA{Buchstaben (\LoesungLuecke{Spalten}{4cm}) und Zahlen (\LoesungLuecke{Zeilen}{4cm})}.
%
Bekannte Tabellenkalkulationsprogramme sind z.B. Microsoft Excel, LibreOffice Calc oder Google Spreadsheets. }